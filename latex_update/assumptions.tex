\section{Assumptions}
\subsection*{Assumption 1: Participants will sleep in transport}
To maximize productivity, our team recommends the participants to sleep in transport, and we assume they comply. The average human requires 5-10 minutes to fall asleep. For this problem, we assume each participant takes 10 minutes fall asleep. 

\subsection*{Assumption 2: Minimizing time zone crossing and incomplete sleep cycles should be prioritized}
D.B. Boivin found that crossing times zones and incomplete sleep cycles are two imposing issues with aerial transport. Thus, we fulfill two conditions:
\begin{enumerate}
	\item Attempt to minimize the number of time zones each participant crosses.
	\item Humans go through a single sleep cycle of 100 minutes, and scientists recommend that humans sleep in complete sleep cycles. So, total sleep time (- 10 minutes  should be a multiple of 100 minutes.
\end{enumerate}

\subsection*{Assumption 3: Plane speed is 900 mph}
Average commercial airliner speed is 500 knots. We assume average speed of our planes is 900 mph. We assume the weather will permit this.

\subsection*{Assumption 4: Participants will not consume treatments for jet lag provided by airliners}
Several treatments for jet lag, such as individual schedule management, different hypnotic agents, other chemicals, exist (Choy \& Salbu, 2011). However, we will disregard these possible treatments for two reasons: 
\begin{enumerate}
	\item Regardless of treatment, we want to minimize any possible effects of jet lag. 
	\item We do not have complete control over the participants. We assume that IMMC cannot force them to take specific treatment for jet lag. 
\end{enumerate}

\subsection*{Assumption 4: The Earth is perfectly spherical}
The largest irregularities on the Earth’s surface are less than 0.1i\% of the Earth’s radius. We assume the surface is a perfect sphere.

