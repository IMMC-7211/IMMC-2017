\section{Assumptions}
\subsection*{Assumption 1: Participants will not be treated for jet lag}
Several treatments for jet lag, such as individual schedule management, hypnotic agents, and chemicals exist (Choy \& Salbu, 2011). However, we disregard these treatments.
\textit{Justification}: Regardless of treatment, we must minimize the effects of jet lag.  
\subsection*{Assumption 2: The Earth is perfectly spherical}
\textit{Justification}: We assume that irregularities in the earth’s structure will not change our results. The largest irregularities on the earth’s surface are less than 0.1\% of the earth’s radius.
\subsection*{Assumption 3: Jet lag is due to geographical change}
Primary cause of jet lag is the change in geographical location rather than political time zone.\\ 
\textit{Justification}: The oscillators in our brain which determine circadian rhythm are affected by geographical and environmental factors, (e.g. hours of sunlight).
\subsection*{Assumption 4: Mean flight cost is calculated by compounding all annual flight data}

\subsection*{Assumption 5: Flights are always available between any two airports}
\textit{Justification}: As test inputs do not give year of meeting, finding actual flight data is impossible, and for ease of computation, we assume that flights are always available.
